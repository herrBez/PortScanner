\chapter{Port Scan Detection}
\label{sec:detector}
Port Scan Detection is fundamental in order
to identify potential attackers or to recognize intrusion attempts.

There are many approaches to detect port scans, but the main idea
is always the same: One target is potentially port scanned if it receives
from the same source a lot of packets in a short time range.
Simple examples of port scan detectors are given by 
Openwall~\cite{scanlogd} and Sophos~\cite{sophos}.
We chose the second method because it is described in more details and the document found
is more recent.






\section{Port Scan Detector}


\subsection{Theory}
According to the description of Sophos their port scan detector works as follows~\cite{sophos}:
\begin{quote}
	A port scan is detected when a detection score of 21 points
	in a time range of 300ms for one individual source IP address is exceeded.
\end{quote}
When a packet from one source is received the detection score of this source
is actualised adding the points accordingly to the following rules:
\begin{itemize}
	\item Destination port $<$ 1024: 3 points
	\item Destination port $>=$ 1024: 1 point
	\item Destination port 11, 12, 13, 2000: 10 points
\end{itemize}
The assignments of the points could seem random, but it can be explained by observing the services running on the different ports and the threats associated to the ports.
\subsubsection{Score explanation}
The Internet Engineering Task Force (IETF) distinguish three ranges of ports: System Ports (0-1023), 
User Ports (1024-49151) and Dynamic and/or Private Ports (49152-65535)~\cite{rfc6335}.
The first range is of particular interest for attackers because it contains many well-known services such as FTP, SSH and HTTP.
The other ranges are generally of less interest.

The Ports 11, 12, 13 and 2000 are of particular interest because they are associated with different attacks or can be used
to gather information about the victim.
According to IANA~\cite{IANAPORTS} the port 11 is assigned to the service Active Users. If this service is running and
receives an UDP or TCP packet it replies with the list of the active users (i.e. the logged users) independently from th
A port scan is detected when a detection score of 21 points
in a time range of 300ms for one individual source IP address is exceeded.
\end{quote}
When a packet from one source is received the detection score of this source
is actualised adding the points accordingly to the following rules:
\begin{itemize}
	\item Destination port $<$ 1024: 3 points
	\item Destination port $>=$ 1024: 1 point
	\item Destination port 11, 12, 13, 2000: 10 points
\end{itemize}
The assignments of the points could seem random, but it can be explained by observing the services running on the different ports and the threats associated to the ports.
\subsubsection{Score explanation}
The Internet Engineering Task Force (IETF) distinguish three ranges of ports: System Ports (0-1023), 
User Ports (1024-49151) and Dynamic and/or Private Ports (49152-65535)~\cite{rfc6335}.
The first range is of particular interest for attackers because it contains many well-known services such as FTP, SSH and HTTP.
The other ranges are generally of less interest.e content of the packet~\cite{systat}.
Port 13 is assigned to the daytime service. If this service is running and receives a TCP or UDP packet it responds with the actual time of day without considering
the content of the received packet. Different machines responds with different date/time format, so this can be used to fingerprint the machines~\cite{portDetails}.
Port 2000 is officially assigned to the CISCO SCCP service, but it is also famous for 
many Trojans such as Der Spaehr, Remote Explorer 2000 among others~\cite{portDetails}.
Port 12 is not associated to any service and attacks but its importance is probably due to the fact that it resides between port 11 and 13.




\subsection{Details of implementation}
The implementation of the Port Scan Detector is largely based on the use of the tcpdump's pcap library
and on a good example given by tcpdump self~\cite{pcaptcpdump}.
This library is used by tcpdump and wireshark in order to get  the packets on the network and choose 
which packets to sniff and which to ignore by using filter expression such as \lstinline!dst host 192.168.0.1! and \lstinline!dst portrange 1-1024!
or a combination of those rules~\cite{pcapFilterRules}.



In order to start a capture session the program has to create a new pcap object using the \lstinline!pcap_open_live! function:
\begin{lstlisting}[frame= single, language=C, caption={Create a new pcap handler.}, label=lst:open_live]
handle = pcap_open_live(dev, SNAP_LEN, 0, 1000, errbuf);
\end{lstlisting}
Another important step is to compile the filter expression. 
\begin{lstlisting}[frame= single, language=C, caption=Pcap functions called to compile the filter expression and to set it.]
pcap_compile(handle, &fp, filter_expression, 0, net);
pcap_setfilter(handle, &fp);
\end{lstlisting}

A fundamental function is the \lstinline|pcap_loop|. The parameters are the pcap object,
the num of packets (-1 for infinity) a callback function that is called each time a packet
is got and the last is usually set as NULL, but it can be used to pass extra arguments for the
callback function.

\begin{lstlisting}[frame= single, language=C, caption=Pcap functions called to start getting the packets on the network (Error handling omitted).]
pcap_loop(handle, num_packets, got_packet, NULL);
\end{lstlisting}



\section{Demo Description}
In listing \ref{lst:detector:help} the options of the port scan detector are described.


\begin{lstlisting}[frame=single, language=BASH, label=lst:detector:help, caption=Help page of the port scan detector.]
Usage: ./port_scan_detector [options]
Options:

--help
-h	print this help and exit
--save-the-port
-s	save all the tcp ports requested from the potential attackers
--device
-d	specify a device to use (e.g. wlan0). 
you can get those devices using the command ifconfig
--verbose
-v	verbose output
--dst-ip
-i	give a destination IP adress
--max-num-of-packets
-m	maximum number packets to get, after which the program ends
--log-file
-l	Save the output in a .tex file
\end{lstlisting}
If the user does not specify any option the default behaviour is used:
The destination IP is set to 127.0.0.1 and the network interface to lo. Only
the packets sent from localhost can be caught and processed.

If the user set the option \lstinline{-s} 









