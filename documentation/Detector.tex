\chapter{Port Scan Detection}
\label{sec:detector}
The aim of the second part of the project consisted of implementing
a port scan detector. Port Scan Detection is fundamental in order
to detect potential attackers.

\section{Port Scan Detector}
There are many approaches to detect port scanning, but the main idea
is always the same: One target is potentially port scanned if it receives
from the same source a lot of packets in a short time range.
This is the approach used by Openwall~\cite{scanlogd} and
Sophos~\cite{sophos}.
We chose the second method because it is described in more details and the document found
is more recent.
\subsection{Theory}
According to the description of Sophos their port scan detector works in the following way~\cite{sophos}:
\begin{quote}
	A port scan is detected when a detection score of 21 points
	in a time range of 300ms for one individual source IP-Address is exceeded.
\end{quote}
When a packet from one source is received the Detection Score of this source
is actualised adding the points accordingly to the following rules:
\begin{itemize}
	\item TCP Destination port $<$ 1024: 3 points
	\item TCP Destination port $>=$ 1024: 1 point
	\item TCP Destination port 11, 12, 13, 2000: 10 points
\end{itemize}
The assignments of the points could seem random, but it can be explained by observing the services running on the different ports and the threats associated to the ports.
\subsubsection{Score explanation}
The Internet Engineering Task Force (IETF) distinguish three ranges of ports: System Ports (0-1023), 
User Ports (1024-49151) and Dynamic and/or Private Ports (49152-65535).

The first range is of particular interest for attackers because it contains many well-known services such as FTP, SSH and HTTP.
The other ranges are generally of minor interest.

The Ports 11, 12, 13 and 2000 are of particular interest because they are associated with different attacks or can be used
to gather information about a victim.

According to IANA the port 11 is assigned to the service Active Users. If this service
receives an UDP or TCP packet it replies with the list of the active users independently from the content of the packet~\cite{systat}.

Port 13 is assigned to the daytime service. If this service receives a TCP or UDP packet it responds with the actual time of day without considering
the content of the received packet. Different machines responds with different date/time format, so this can be used to fingerprint the machines~\cite{portDetails}.

Port 2000 is officially assigned to the CISCO SCCP service, but it is also well known for 
many Trojans such as Der Spaehr, Remote Explorer 2000 \dots~\cite{portDetails}.

Port 12 is not associated to any service and attacks but is maybe important because it's between port 11 and 13.



\subsection{Details of implementation} 
