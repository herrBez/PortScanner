\chapter{Port Scan Detection}
\label{sec:detector}
Port Scan Detection is fundamental in order
to identify potential attackers or to recognize intrusion attempts.

There are many approaches to detect port scans, but the main idea
is always the same: One target is potentially port scanned if it receives
from the same source a lot of packets in a short time range.
Simple examples of port scan detectors are given by 
Openwall~\cite{scanlogd} and Sophos~\cite{sophos}.
We chose the second method because it is described in more details and the document found
is more recent.






\section{Port Scan Detector}


\subsection{Theory}
\label{subsec:detector:theory}
According to the description of Sophos their port scan detector works as follows~\cite{sophos}:
\begin{quote}
	A port scan is detected when a detection score of 21 points
	in a time range of 300ms for one individual source IP address is exceeded.
\end{quote}
When a packet from one source is received the detection score of this source
is actualised adding the points accordingly to the following rules:
\begin{itemize}
	\item Destination port $<$ 1024: 3 points
	\item Destination port $>=$ 1024: 1 point
	\item Destination port 11, 12, 13, 2000: 10 points
\end{itemize}
The assignments of the points could seem random, but it can be explained by observing the services running on the different ports and the threats associated to the ports.



\subsubsection{Score explanation}
The Internet Engineering Task Force (IETF) distinguish three ranges of ports: System Ports (0-1023), 
User Ports (1024-49151) and Dynamic and/or Private Ports (49152-65535)~\cite{rfc6335}.
The first range is of particular interest for attackers because it contains many well-known services such as FTP, SSH and HTTP.
The other ranges are generally of less interest.


The Ports 11, 12, 13 and 2000 are of particular interest because they are associated with different attacks or can be used
to gather information about the victim.
According to IANA~\cite{IANAPORTS} the port 11 is assigned to the service Active Users. If this service is running and
receives an UDP or TCP packet it replies with the list of the active users (i.e. the logged users) independently from the
content of the packet received~\cite{systat}.
Port 13 is assigned to the daytime service. If this service is running and receives a TCP or UDP packet it responds with the actual time of day without considering
the content of the received packet. Different machines responds with different date/time format, so this can be used to fingerprint the machines~\cite{portDetails}.
Port 2000 is officially assigned to the CISCO SCCP service, but it is also famous for 
many Trojans such as Der Spaehr, Remote Explorer 2000 among others~\cite{portDetails}.
Port 12 is not associated to any service and attacks but its importance is probably due to the fact that it resides between port 11 and 13.




\subsection{Details of implementation}
The implementation of the Port Scan Detector is largely based on the use of the tcpdump's pcap library
and on a good example given by tcpdump~\cite{pcaptcpdump}.
This library is used by tcpdump and wireshark in order to get  the packets on the network.
Pcap gives the possibility to choose which packets to sniff and which to ignore by using filter expression 
such as \lstinline!dst host 192.168.0.1! and \lstinline!dst portrange 1-1024! or a combination of those rules~\cite{pcapFilterRules}.



In order to start a capture session the program has to create a new pcap object using the \lstinline!pcap_open_live! function:
\begin{lstlisting}[style=MyC, caption={Create a new pcap handler.}, label=lst:open_live]
handle = pcap_open_live(dev, SNAP_LEN, 0, 1000, errbuf);
\end{lstlisting}
Another important step is to compile the filter expression and to associate it to the pcap handle.
\begin{lstlisting}[style=MyC, caption=Pcap functions called to compile the filter expression and to set it.]
sprintf(filter_expression, "dst host %s", dst_ip);
pcap_compile(handle, &fp, filter_expression, 0, net);
pcap_setfilter(handle, &fp);
\end{lstlisting}



A fundamental function is \lstinline|pcap_loop|.
The parameters are the pcap object,
the number of packets (0 or a negative number for infinity), a callback function
 that is called each time a packet
is got and the last argument is usually set as NULL, but it can be used to pass extra arguments for the
callback function.
Thanks to this function the program enters in a loop and begins to catch packets and process them using the callback function.
\begin{lstlisting}[frame= single, language=C, caption=Pcap functions called to start getting the packets on the network.]
pcap_loop(handle, num_packets, got_packet, NULL);
\end{lstlisting}
The function \lstinline|got_packets| is the core of the implementation: Inside this function the packets are processed.

\begin{lstlisting}[style=MyC]
void got_packet(u_char *args, const struct pcap_pkthdr *header, 
const u_char *packet) {
	...
	ip = (struct sniff_ip*)(packet + SIZE_ETHERNET);
\end{lstlisting}

The first step is to cast the packet in the structure \lstinline|sniff_ip| defined in the file \path{includes/important_header.h}.
The source address is read from the packet and used to check if the packet comes from a known source. If it is not the 
case the source ip is added to the list and a thread is started.

\begin{lstlisting}[style=MyC]
	actual_node = contains_node(head, actual_adress);
	if(actual_node == NULL){ //First packet of this source ip
		push(&head, actual_adress);
		actual_node = contains_node(head, actual_adress);
		pthread_create(&actual_node->thread, NULL, thread_function, 
						actual_node);
	}
\end{lstlisting}


According to the packet's protocol the counters of the \lstinline|actual_node|
are actualised and in case of a TCP or UDP packet the function \lstinline|process_tcp| and 
\lstinline|process_udp| are called. Because the UDP callback function is very similar to the 
TCP one only the latter one is described.
\begin{lstlisting}[style=MyC, caption=The \lstinline|got_packet| function.]
	switch(ip->ip_p) {
		case IPPROTO_TCP:
			actual_node->tcp[INDEX_TCP]++;
			process_tcp(packet, ip, size_ip, actual_node);
			break;
		case IPPROTO_UDP:
			actual_node->udp++;
			process_udp(packet, ip, size_ip, actual_node);
			break;
		case IPPROTO_ICMP:
			actual_node->icmp++;
			break;
		case IPPROTO_IP:
			actual_node->ip++;
			break;
		default:
			actual_node->unknown++;
			break;
	} 
\end{lstlisting}
The packet is cast to TCP, the destination port is read and 
the score for the port is saved in the variable \lstinline|tmp|. 
\begin{lstlisting}[style=MyC]
void process_tcp(const u_char *packet, const struct sniff_ip *ip,
	int size_ip, node_t * actual_node){
	...
	tcp = (struct sniff_tcp*)(packet + SIZE_ETHERNET + size_ip);
	...
	dst_port = ntohs(tcp->th_dport);
	int tmp = get_score(dst_port);
\end{lstlisting}
The score is actualized using the principle described in
section \ref{subsec:detector:theory}
\begin{lstlisting}[style=MyC]
int get_score(int dst_port){
	if((dst_port >= 11 && dst_port <= 13) || dst_port == 2000)
		return 10;
	else if(dst_port < 1024)
		return 3;
	else if(dst_port >= 1024)
		return 1;
	return 0;
}
\end{lstlisting}



Depending on the flags
the type of TCP packet is recognized and both the counters for the different
packets and the score for the different scan types are actualized.
\begin{lstlisting}[style=MyC]
	if(tcp->th_flags == get_syn_scan_flags()){
		index = INDEX_SYN; printf("SYN\n");
	} else if(tcp->th_flags == get_ack_scan_flags()){
		index = INDEX_ACK; printf("ACK\n");
	} else if(tcp->th_flags == get_null_scan_flags()){
		index = INDEX_NULL; printf("NULL\n");
	} else if(tcp->th_flags == get_xmas_scan_flags()){
		index = INDEX_XMAS; printf("XMAS\n");
	} else if(tcp->th_flags == get_fin_scan_flags()){
		index = INDEX_FIN; printf("FIN\n");
	} else if(tcp->th_flags == get_maimon_scan_flags()){
		index = INDEX_MAIMON; printf("MAIMON (FIN | ACK)\n");
	}
	else {
		index = INDEX_UNKNOWN;
		printf("Not known scan type (Flag set to 0x%X := ", tcp->th_flags);
		print_tcp_flags(tcp->th_flags);
		printf(")\n");
	}
	actual_node->tcp_actual_score[INDEX_TCP] += tmp;
	actual_node->tcp_total_score[INDEX_TCP] += tmp;
	actual_node->tcp_actual_score[index] += tmp;
	actual_node->tcp_total_score[index] += tmp;
}
\end{lstlisting}

As already mentioned for each different source IP a thread is started.
This thread waits 300 milliseconds and then check the actual score for
the different TCP/UDP scan types. If the actual score is bigger than
21 a port scan is detected and a briefly message is output. 
\begin{lstlisting}[style=MyC]
void * thread_function(void * _node){	
	node_t * n = (node_t *) _node;
	while(!break_thread){
		nanosleep(&_my_time, NULL); //Wait 300 ms
		controlScore(n);
	}
	return NULL;
}

void controlScore(node_t * n){
	int i;
	int tot_detected = 0;
	for(i = 0; i < INDEX_SIZE; i++){
		if(n->tcp_actual_score[i] >= PORT_SCAN_SCORE){
			printf("*** %7s TCP SCAN FROM %s dectected\n",
						index_to_string(i), n->ip_src);
			n->tcp_scan_detected[i]++;
			tot_detected++;
		}
		n->tcp_actual_score[i] = 0;
	}
	if(n->udp_actual_score >= PORT_SCAN_SCORE){
		printf("=== UDP SCAN FROM %s ===\n", n->ip_src);
		n->udp_scan_detected++;
	}
	n->udp_actual_score = 0;
}
\end{lstlisting}


In order to stop the loop the function \lstinline|pcap_breakloop()| has to be called.
In the implementation this function is called in the SIGINT (Ctrl-C) handler \lstinline|my_sigint_handler()| as shown
in listing \ref{lst:detector:sigint}. This function stop the loop and cause the stop of all the threads.
\begin{lstlisting}[style=MyC, caption=Handler that process the SIGINT signal., label=lst:detector:sigint]
void my_sigint_handler(int d){
	pcap_breakloop(handle);
	break_thread = true;
}
\end{lstlisting}
After all the threads are joined, a summary is printed and all resources are freed.



\section{Demo Description}
In listing \ref{lst:detector:help} the options of the port scan detector are described.
\begin{lstlisting}[frame=single, language=BASH, label=lst:detector:help, caption=Help page of the port scan detector.]
Usage: ./port_scan_detector [options]
Options:

--help
-h	print this help and exit
--save-the-port
-s	save all the tcp ports requested from the potential attackers
--device
-d	specify a device to use (e.g. wlan0). 
	  you can get those devices using the command ifconfig
--verbose
-v	verbose output
--dst-ip
-i	give a destination IP adress
--max-num-of-packets
-m	maximum number packets to get, after which the program ends
--log-file
-l	Save the output in a .tex file
\end{lstlisting}
If the user does not specify any option the default behaviour is used:
The destination IP is set to 127.0.0.1 and the network interface to lo. Only
the packets sent from localhost can be caught and processed.

If the user set the option \lstinline|-s| the program takes track of how many times
each potential attacker which UDP or TCP port visit.

Thanks to the option \lstinline|-m| the user can set the maximum number of packets to get, after which the programs end.

The option \lstinline|-l| save all the information in a \LaTeX{} file.

A known issue with the program is that if the user set the option \lstinline|-i| he or she have also to set the option \lstinline|-d|.

After the start of the program the information about the single packets and the scan detected are printed on the terminal. The program can be ended either by clicking Ctrl-C or by waiting the reach of the num packets (It can happen only if the option m is set or an error occurred). If the program is ended gracefully it prints a summary of what happened
during the capture session.









