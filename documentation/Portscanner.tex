\documentclass[a4paper]{scrreprt}
%\RequirePackage[a4paper,left=2.5cm,right=2.5cm,top=2.5cm,bottom=3cm]{geometry}
\usepackage[utf8]{inputenc}
\usepackage{hyperref}
\hypersetup{colorlinks = true, pdfborderstyle = {/S/U/W 1}, linkcolor = black}
%\usepackage{tabularx}
%\usepackage{todonotes}
\usepackage{listings}
\lstset{ 
	captionpos=b,
	basicstyle=\small\ttfamily,
	tabsize=2,
}
%\usepackage[backend=bibtex]{biblatex}
%\usepackage{fullpage}
%\bibliography{Portscanner}

	
\begin{document}

\title{Portscanner}
%\subtitle{Applied Information Security}
\author{Mirko Bez \and Simon Targa} 
\date{\today}
\maketitle
\tableofcontents
\newpage



\chapter*{Introduction}
The first goal of the project was to understand the theory behind port scanners and their detection. The next goal was to implement a port scanner. The scanner should be written in C and support various scan methods (e.g.\ TCP connect scan, TCP SYN scan \dots). The last goal
was to implement our own port scan detector. 

The final result consists of two programs: a port scanner and a port scan detector.
The port scanner tries to simulate the behavior of nmap, which is one of the most used
programs for port scanning. The port scan detector uses the pcap library in order to sniff
the incoming network packets to recognize port scan attempts.

This document describes how the two programs work and the theory behind them.
The focus of section~\ref{sec:connect} is on the TCP connect scan which is the most simple port scan technique. 
Section~\ref{sec:syn} is about how the TCP SYN scan works and how it was implemented within the scope of this project. Section~\ref{sec:xmas} describes the scan methods Xmas, TCP NULL and Fin scan and their
implementations. Section~\ref{sec:detector} is dedicated on how port scanning attempts can be detected and blocked by an IT administrator.

\section*{Motivation}
Port scanners are used to determine which ports are open. This information can be used by
attackers to identify services running on a host and exploit vulnerabilities. 


For example, researchers recently identified bugs in Oracle’s Java SE that allow arbitrary execution of code, access to security sensitive data, unauthorized changes in security configurations, and so on~\cite{dangerports}.



\chapter{Port Scannner}
\section{TCP connect scan}
\label{sec:connect}
	\subsection{Theory}
	The TCP connect scan is probably the most easy method to scan for open ports. It simply takes advantage of the system call
	\emph{connect} of the underlying operating system, in order to establish a connection with the target machine and port. Afterwards the
	returned value of the system call is used to determine if the port to check is either closed or open at the target machine~\cite{nmap2009}.
	
	
	\subsection{Details of implementation}
	In order to use the system call~\lstinline{connect} the implementation uses C sockets of the type \lstinline|SOCK_STREAM|. This type
	of socket allows us, to establish a tcp connection to the target machine.
	\begin{lstlisting}[frame= single, language=C, caption=C code to create a tcp socket in C]
	int mysocket;
	mysocket=socket(AF_INET, SOCK_STREAM, 0);
	\end{lstlisting}
	Additionally to the socket we also have to use a structure of the type \lstinline|sockaddr_in| to connect to the target machine and port.
	The structure is needed to define the ip address of the target machine and the port to use for the connection. The listing shows the
	code of how to assign the ip address and port to a structure of the type \lstinline|sockaddr_in|. 
	\begin{lstlisting}[frame= single, language=C, caption= C code to use the structure \lstinline|sockaddr_in|]
	struct sockaddr_in server;
	struct hostent *host;
	hostname = gethostbyname(p->host_name);
	memcpy( (char *)&server.sin_addr, host->h_addr_list[0], host->h_length);
	server.sin_family = AF_INET;
	server.sin_port = htons(port);
	\end{lstlisting}
	The last step is to use the created socket and  \lstinline|sockaddr_in| structure to connect to the target machine and port. If the 
	connection could be established we know that the port is open. To check if the connection attempt was successful, we only have to
	check the return value of the connect() function. Upon successful completion, connect() shall return 0. The code to use the connect() function is shown in the listing.
	\begin{lstlisting}[frame= single, language=C, caption= C code to use the connect() to check if port is open]
	if(connect(mysocket, (struct sockaddr *)&server, sizeof(server))>=0){
		printf("TCP - Port %d is open\n", i);
		close(mysocket);
		mysocket = socket(AF_INET, SOCK_STREAM, 0);
	}
	\end{lstlisting}
	
	\section{TCP SYN scan}
	\label{sec:syn}
	\subsection{Theory}
	
	
	
	\subsection{Details of implementation}
	In order to only send a syn request instead of open a full tcp connection (including handshake) the implementation uses raw sockets. Raw sockets allow to control
	every section of the packets that will be sent. The function socket(), as shown in listing, can be used to create a raw socket that uses the tcp protocol.
	\begin{lstlisting}[frame= single, language=C, caption= C code to use the connect() to check if port is open]
		int mysocket;
		mysocket=socket(AF_INET, SOCK_RAW,  IPPROTO_TCP);
	\end{lstlisting}
	Before we can send a syn request to the target machine, we have to build the packet to be sent. To send packets with a raw socket the function sendto() is used.
	It's second parameter is a pointer to the message to be sent, which 
	is the packet that we build. It consists of the tcp theader, the ip header and the data
	to be sent. As we only want to send a syn request we don't care about the data, therefore it is empty. We start building the packet by filling in the IP-Header. We don't need optional fields therefore the we use the minimal size possible size of the ip header which is 160 Bits (5*32 Bits). We use the ip version 4, which is still the most widely used ip version. The length of our packet
	is the sum of the length of the ip header and the length of the tcp header. For the time to live we choose 64, which should be big enough
	fur our purpose. As transfer protocol we set the tcp protocol. The source address of the ip header is set to the ip address of the scanning system and the destination address is set to the address of the target system to scan. To have a complete ip header we also
	have to calculate its check sum.
	\begin{lstlisting}[frame= single, language=C, caption= C code to fill in ip header]
	//Fill in the IP Header
	iph->ihl = 5;
	iph->version = 4;
	iph->tos = 0;
	iph->tot_len = sizeof (struct ip) + sizeof (struct tcphdr);
	iph->id = htons (54321); //Id of this packet
	iph->frag_off = htons(16384);
	iph->ttl = 64;
	iph->protocol = IPPROTO_TCP;
	iph->saddr = inet_addr ( source_ip );   
	iph->daddr = dest_ip.s_addr;
	iph->check = csum(datagram, iph->tot_len>>1);
	\end{lstlisting}
	Before the packet can be sent, we also need to fill in the tcp header. In order to send a syn request we only set the syn flag to true and all the other flags to false.
	\begin{lstlisting}[frame= single, language=C, caption= C code to set flags in tcp header]
	tcph->fin=0;
	tcph->syn=1;
	tcph->rst=0;
	tcph->psh=0;
	tcph->ack=0;
	tcph->urg=0;
	\end{lstlisting}
	The last step before we can send the packet is to set the destination port (the port to scan)  in the tcp header and calculate its check sum.
	\begin{lstlisting}[frame= single, language=C, caption= C code to set port and calculate checksum in tcp header]
	tcph->dest = htons ( port );
	tcph->check = csum(&psh, sizeof(struct pseudo_header))
	\end{lstlisting}
	The function sendto() is used to send the created packet to the target machine and port. If the sending fails the program terminates with
	an error, because then we cannot scan for open ports.
	\begin{lstlisting}[frame= single, language=C, caption= C code to set port and calculate checksum in tcp header]
	if(sendto(s, datagram, packetsize, 0 , &dest, destsize)< 0)
	{
	perror("Error sending packet: ");
	exit(0);
	}
	\end{lstlisting}
	To complete the syn port scan, we also have to receive the answer to our sent packet. To receive packets with from a raw socket the function recvfrom() is used. The function call blocks, until it
	receives a packet from the given socket. Therefore we used the function select, to add an timer to the receiving socket. As you can see in the code of the following listing, we add an timer of 1 sec
	to the receiving socket and only use the recvfrom() function if the socket contains a packet. If we cannot receive an answer then we simply scan the next port in our implementation.
	\begin{lstlisting}[frame= single, language=C, caption= C code to receive the packet]
 FD_ZERO(&fds);
 FD_SET(s, &fds);
 tv.tv_sec = 1;
 tv.tv_usec = 0;
 select(s+ 1, &fds, NULL, NULL, &tv);		
 if (FD_ISSET(s, &fds))
 {
 data_size = recvfrom(s, buffer, 65536, 0, &saddr, &saddr_size);
	................................
 }else{
  printf("Timeout, port %d filtered by firewall", port);
  return 0;
 }
	\end{lstlisting}
	Because it could be the case that we receive packets from other requests, we first have to check, if the received packet
	is an answer to our request. To do so we use the IF-Statement of the Listing~\ref{lst:checkorigin}. It checks if the source port of the received packet equals the destination port of our sent packet and
	if the source ip equals to the destination ip to which we sent the packet.
	\begin{lstlisting}[frame= single, language=C, caption= IF statement to check origin of packet, label=lst:checkorigin]
	if(source.sin_addr.s_addr == dest_ip.s_addr &&
	port == ntohs(tcph->source))
	\end{lstlisting}
	If the received packet passes the check, we know that we have the packet we were looking for. To test if the port is open we finally only have to check if the ack and syn flags are set in the tcp header
	of the answer. To do so, we first extract the tcp header from our answer by using the length of the ip header as an offset. This works because the first bytes of our answer contain the ip header, which
	is followed by the tcp header. As it can be seen in the listing~\ref{lst:synack}, we finally use an IF statement to check if the tcp header contains the flags which we desire. If it is the case, we know that the scanned
	port is open.
	\begin{lstlisting}[frame= single, language=C, caption=C code to check if answer contains syn and ack flag, label=lst:synack]
	struct tcphdr *tcph=(struct tcphdr*)(buffer + iphdrlen);
	
	if(tcph->syn == 1 && tcph->ack == 1){
	printf("Port: %d is open\n", port);
	} 
	\end{lstlisting}
	\section{XMAS, TCP NULL and FIN scan}
		\label{sec:xmas}
		\subsection{Theory}
		
		\subsection{Details of implementation}
		The implementation of the XMAS, TCP NULL and FIN scan is quiet similar to the implementation of the syn scanner. In fact these four scan methods have very much in common.
		They all need a raw socket to work. There are only 2 main differences between this scan methods and the syn scan. The first one is that XMAS, NULL and FIN scan set different flags
		in the tcp header. As the name suggests the NULL scan sets none of the flags and the FIN scan only sets the FIN flag. The XMAS scan sets the FIN, PSH, and URG flags, lighting the packet up like a Christmas tree.
		The second difference of this three scan methods to the syn method is how the answer to the request is used to determine if a port is open or not. If we get a packet with the RST flag as an answer of one
		of this three scan methods, we know the port is closed. If we don't get an answer we know that the server must have dropped the packet because of an illegal request (RFC 793).
		
		

\chapter{Port Scan Detection}
\label{sec:detector}
Port Scan Detection is fundamental in order
to identify potential attackers or to recognize intrusion attempts.

There are many approaches to detect port scans, but the main idea
is always the same: One target is potentially port scanned if it receives
from the same source a lot of packets in a short time range.
Simple examples of port scan detectors are given by 
Openwall~\cite{scanlogd} and Sophos~\cite{sophos}.
We chose the second method because it is described in more details and the document found
is more recent.






\section{Port Scan Detector}


\subsection{Theory}
According to the description of Sophos their port scan detector works as follows~\cite{sophos}:
\begin{quote}
	A port scan is detected when a detection score of 21 points
	in a time range of 300ms for one individual source IP address is exceeded.
\end{quote}
When a packet from one source is received the detection score of this source
is actualised adding the points accordingly to the following rules:
\begin{itemize}
	\item Destination port $<$ 1024: 3 points
	\item Destination port $>=$ 1024: 1 point
	\item Destination port 11, 12, 13, 2000: 10 points
\end{itemize}
The assignments of the points could seem random, but it can be explained by observing the services running on the different ports and the threats associated to the ports.
\subsubsection{Score explanation}
The Internet Engineering Task Force (IETF) distinguish three ranges of ports: System Ports (0-1023), 
User Ports (1024-49151) and Dynamic and/or Private Ports (49152-65535)~\cite{rfc6335}.
The first range is of particular interest for attackers because it contains many well-known services such as FTP, SSH and HTTP.
The other ranges are generally of less interest.

The Ports 11, 12, 13 and 2000 are of particular interest because they are associated with different attacks or can be used
to gather information about the victim.
According to IANA~\cite{IANAPORTS} the port 11 is assigned to the service Active Users. If this service is running and
receives an UDP or TCP packet it replies with the list of the active users (i.e. the logged users) independently from th
A port scan is detected when a detection score of 21 points
in a time range of 300ms for one individual source IP address is exceeded.
\end{quote}
When a packet from one source is received the detection score of this source
is actualised adding the points accordingly to the following rules:
\begin{itemize}
	\item Destination port $<$ 1024: 3 points
	\item Destination port $>=$ 1024: 1 point
	\item Destination port 11, 12, 13, 2000: 10 points
\end{itemize}
The assignments of the points could seem random, but it can be explained by observing the services running on the different ports and the threats associated to the ports.
\subsubsection{Score explanation}
The Internet Engineering Task Force (IETF) distinguish three ranges of ports: System Ports (0-1023), 
User Ports (1024-49151) and Dynamic and/or Private Ports (49152-65535)~\cite{rfc6335}.
The first range is of particular interest for attackers because it contains many well-known services such as FTP, SSH and HTTP.
The other ranges are generally of less interest.e content of the packet~\cite{systat}.
Port 13 is assigned to the daytime service. If this service is running and receives a TCP or UDP packet it responds with the actual time of day without considering
the content of the received packet. Different machines responds with different date/time format, so this can be used to fingerprint the machines~\cite{portDetails}.
Port 2000 is officially assigned to the CISCO SCCP service, but it is also famous for 
many Trojans such as Der Spaehr, Remote Explorer 2000 among others~\cite{portDetails}.
Port 12 is not associated to any service and attacks but its importance is probably due to the fact that it resides between port 11 and 13.




\subsection{Details of implementation}
The implementation of the Port Scan Detector is largely based on the use of the tcpdump's pcap library
and on a good example given by tcpdump self~\cite{pcaptcpdump}.
This library is used by tcpdump and wireshark in order to get  the packets on the network and choose 
which packets to sniff and which to ignore by using filter expression such as \lstinline!dst host 192.168.0.1! and \lstinline!dst portrange 1-1024!
or a combination of those rules~\cite{pcapFilterRules}.



In order to start a capture session the program has to create a new pcap object using the \lstinline!pcap_open_live! function:
\begin{lstlisting}[frame= single, language=C, caption={Create a new pcap handler.}, label=lst:open_live]
handle = pcap_open_live(dev, SNAP_LEN, 0, 1000, errbuf);
\end{lstlisting}
Another important step is to compile the filter expression. 
\begin{lstlisting}[frame= single, language=C, caption=Pcap functions called to compile the filter expression and to set it.]
pcap_compile(handle, &fp, filter_expression, 0, net);
pcap_setfilter(handle, &fp);
\end{lstlisting}

A fundamental function is the \lstinline|pcap_loop|. The parameters are the pcap object,
the num of packets (-1 for infinity) a callback function that is called each time a packet
is got and the last is usually set as NULL, but it can be used to pass extra arguments for the
callback function.

\begin{lstlisting}[frame= single, language=C, caption=Pcap functions called to start getting the packets on the network (Error handling omitted).]
pcap_loop(handle, num_packets, got_packet, NULL);
\end{lstlisting}



\section{Demo Description}
In listing \ref{lst:detector:help} the options of the port scan detector are described.


\begin{lstlisting}[frame=single, language=BASH, label=lst:detector:help, caption=Help page of the port scan detector.]
Usage: ./port_scan_detector [options]
Options:

--help
-h	print this help and exit
--save-the-port
-s	save all the tcp ports requested from the potential attackers
--device
-d	specify a device to use (e.g. wlan0). 
you can get those devices using the command ifconfig
--verbose
-v	verbose output
--dst-ip
-i	give a destination IP adress
--max-num-of-packets
-m	maximum number packets to get, after which the program ends
--log-file
-l	Save the output in a .tex file
\end{lstlisting}
If the user does not specify any option the default behaviour is used:
The destination IP is set to 127.0.0.1 and the network interface to lo. Only
the packets sent from localhost can be caught and processed.

If the user set the option \lstinline{-s} 












\chapter{Future Work}
There is room of improvement in our project. Section~\ref{sec:improve:scan} gives some advice how the port scanner may be improved and section~\ref{sec:improve:detector} suggests some 
possible improvements for the port scan detector. 
\section{Port scanner}
\label{sec:improve:scan}
\begin{itemize}
\item Currently in our port scanner we have set a timeout to wait one second for a response. This value could be increased/decreased by
doing some tests on different machines and servers to determine a stable value for the timeout.
\item Our port scanner loops over every port to scan and only tries once to send a request to the server. If a server is unstable this behavior could
led to inconsistent results. It may be better to choose a number of repetition to send the request after getting the timeout.
\item Multithreading could be used to scan different ports simultaneously. 
\item Other scan methods could be added the port scanner (e.g. Idle Scan, UDP scan).
\end{itemize}
\section{Port scan Detector}
\label{sec:improve:detector}
\begin{itemize}
\item The port scan detector currently uses only one network interface. It may be also interesting to have an option to use more network interfaces simultaneously.
\item Currently the detector takes the IP address of the interface as an command line argument. In make it more user-friendly the IP address could be determined at runtime.
\item In order to distinguish between a TCP connect and SYN scan techniques could be implemented to recognizes a complete handshake.
\item Implementation of other port scan detection approaches.
	

\end{itemize}



\section{Extensions}


\newpage
%\printbibliography

\end{document}
