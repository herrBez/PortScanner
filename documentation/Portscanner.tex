\documentclass[a4paper]{scrreprt}
\RequirePackage[a4paper,left=2cm,right=2cm,top=2cm,bottom=3cm]{geometry}
\usepackage[utf8]{inputenc}
\usepackage{hyperref}
\hypersetup{colorlinks = true, pdfborderstyle = {/S/U/W 1}, linkcolor = black}
\usepackage{tabularx}
%\usepackage{todonotes}
\usepackage{listings}
\lstset{ 
	captionpos=b,
	basicstyle=\small\ttfamily,
	tabsize=2,
}
%\usepackage[backend=bibtex]{biblatex}
%\usepackage{fullpage}
\bibliography{Portscanner}

	
\begin{document}

\title{Portscanner}
%\subtitle{Applied Information Security}
\author{Mirko Bez \and Simon Targa} 
\date{\today}
\maketitle
\tableofcontents
\newpage



\section*{Introduction}
The first goal of the project was to understand the theory behind port scanners and their detection. The next goal was to implement a port scanner. The scanner should be written in C and support various scan methods (e.g.\ TCP connect scan, TCP SYN scan \dots). The last goal
was to implement our own port scan detector. 

The final result consists of two programs: a port scanner and a port scan detector.
The port scanner tries to simulate the behavior of nmap, which is one of the most used
programs for port scanning. The port scan detector uses the pcap library in order to sniff
the incoming network packets to recognize port scan attempts.

This document describes how the two programs work and the theory behind them.
The focus of section~\ref{sec:connect} is on the TCP connect scan which is the most simple port scan technique. 
Section~\ref{sec:syn} is about how the TCP SYN scan works and how it was implemented within the scope of this project. Section~\ref{sec:xmas} describes the scan methods Xmas, TCP NULL and Fin scan and their
implementations. Section~\ref{sec:detector} is dedicated on how port scanning attempts can be detected and blocked by an IT administrator.

\section*{Motivation}
Port scanners are used to determine which ports are open. This information can be used by
attackers to identify services running on a host and exploit vulnerabilities. 


For example, researchers recently identified bugs in Oracle’s Java SE that allow arbitrary execution of code, access to security sensitive data, unauthorized changes in security configurations, and so on~\cite{dangerports}.

\input{Scanner}

\chapter{Port Scan Detector}
\label{sec:detector}



\newpage
%\printbibliography

\end{document}
