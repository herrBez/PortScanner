\documentclass[a4paper]{scrartcl}
\usepackage{fullpage}
\author{Simon Targa \and Mirko Bez}
\title{Port Scan Detector Output File}
\usepackage{hyperref}% filename: prova.tex
\begin{document}
\maketitle{}
\tableofcontents
\newpage
\section{List of potential attackers}
\emph{ICMP, UDP, IP, UK (Unknown), TCP indicate the number of the packets of that type received.} \\ \\ 
\begin{tabular}{| c | c | c | c | c | c | c | c | c | }
	\hline
	ID & IP Address & First & Last  & ICMP & UDP & IP & UK & TCP \\ 
   &            & Packet & Packet &     &    &    &     & \\ 
	\hline
	1 & 127.0.0.1 & 11.01.16 15:58:27 & 11.01.16 15:58:34 &   0 &   0 &   0 &   0 & 10240 \\ 
	\hline
\end{tabular}
\section{TCP Details}
\emph{SYN, FIN, XMAS, NULL, ACK, UK (Unkown) indicate the number of the tcp packets of that type received.} \\ \\ 
\begin{tabular}{| c | c | c | c | c | c | c | c | c | c | c | }
\hline 
ID & IP & TCP & SCAN & SYN  & FIN & XMAS & NULL & ACK & MAIMON & UK \\ 
 &  & & DETECTED &   &  &  &  &  & &  \\ 
\hline 
1 & 127.0.0.1 & 10240 &   9 &   0 & 3072 &   0 &   0 &   0 & 2048 & 5120\\ \hline
\end{tabular}
\subsection{Type of TCP SCAN DETECTED}
\begin{tabular}{| c | c | c | c | c | c | c | c | c | }
\hline 
IP-Address &  SYN & XMAS & ACK & MAIMON & NULL & FIN & UK & TCP \\ 
\hline 
127.0.0.1& 0 & 0 & 0 & 4 & 0 & 5 & 9 & 9  \\ 
 \hline
\end{tabular}
\section{UDP Details}
\begin{tabular}{| c | c | c | c |}
\hline 
ID & IP-Address & SCAN     & TOTAL \\ 
   &            & DETECTED & SCORE \\ 
\hline
1 & 127.0.0.1 & 0 & 0 \\ \hline 
\end{tabular}
\section{Summary}
\emph{List containing the result and some meta data} \\ \\ 
\begin{tabular}{| l | r |}
\hline
Filter Expression & dst host 127.0.0.1 \\ 
Device used & lo \\ 
Number of TCP scan detected & 9 \\ 
Number of different scanners & 1 \\ 
Number of different sources (i.e. \# potential attackers) & 1 \\ 
Number of received packets &  10241 \\ 
Scan begin & Mon Jan 11 15:58:23 2016
 \\ 
Scan end & 	Mon Jan 11 15:58:35 2016
 \\ 
Total Elapsed time & 12 seconds \\ 
\hline
\end{tabular}
\end{document}
